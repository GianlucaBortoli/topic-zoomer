\documentclass{sig-alternate-05-2015}
\usepackage{subscript}
\usepackage{tikz}

\begin{document}

\title{Topic-zoomer}

\numberofauthors{2}
\author{
\alignauthor
Gianluca Bortoli\\
       \affaddr{DISI-University of Trento}\\
       \affaddr{Via Sommarive, 9}\\
       \affaddr{38123 Trento, Italy}\\
       \email{gianluca.bortoli@studenti.unitn.it}
\alignauthor
Pierfrancesco Ardino\\
       \affaddr{DISI-University of Trento}\\
	   \affaddr{Via Sommarive, 9}\\
       \affaddr{38123 Trento, Italy}\\
       \email{pierfrancesco.ardino@studenti.unitn.it}
}
\maketitle

\begin{abstract}
Abstract
\cite{*} % little hack
\end{abstract}

\begin{CCSXML}
	<ccs2012>
	<concept>
	<concept_id>10002951.10003317.10003347.10003350</concept_id>
	<concept_desc>Information systems~Recommender systems</concept_desc>
	<concept_significance>500</concept_significance>
	</concept>
	<concept>
	<concept_id>10002951.10003317.10003318.10003321</concept_id>
	<concept_desc>Information systems~Content analysis and feature selection</concept_desc>
	<concept_significance>300</concept_significance>
	</concept>
	<concept>
	<concept_id>10002951.10003317.10003359.10003360</concept_id>
	<concept_desc>Information systems~Test collections</concept_desc>
	<concept_significance>300</concept_significance>
	</concept>
	<concept>
	<concept_id>10002951.10003317.10003359.10003362</concept_id>
	<concept_desc>Information systems~Retrieval effectiveness</concept_desc>
	<concept_significance>300</concept_significance>
	</concept>
	</ccs2012>
\end{CCSXML}

\ccsdesc[500]{Information systems~Recommender systems}
\ccsdesc[300]{Information systems~Content analysis and feature selection}
\ccsdesc[300]{Information systems~Test collections}
\ccsdesc[300]{Information systems~Retrieval effectiveness}

\printccsdesc

\keywords{Data mining}


\section{Introduction}
Intro

\section{Data Collection and \\pre-processing pipeline}
Data collection

\section{Experiments}
Experiments

\section{Conclusions}
Conclusions

%\end{document}  % This is where a 'short' article might terminate
%
% The following two commands are all you need in the
% initial runs of your .tex file to
% produce the bibliography for the citations in your paper.
\bibliographystyle{abbrv}
\bibliography{biblio}% sigproc.bib is the name of the Bibliography in this case
% You must have a proper ".bib" file
%  and remember to run:
% latex bibtex latex latex
% to resolve all references
%
% ACM needs 'a single self-contained file'!
%

\end{document}
